\subsection{{\tt panelmodel}: Panel Data (TSCS) Methods for [Continuous] Dependent Variables}
\label{zelig2plm}
The \texttt{plm} package allows for the estimation of panel/CSTS (cross-section, time series) models in a form familiar to political scientists, economists, and sociologists using such data. As the \texttt{plm} packages authors state, ``plm is a package for R which intends to make the estimation of linear panel models straightforward. plm provides functions to estimate a wide variety of models and to make (robust) inference.''

The package provides functions to estimate linear models for panel data with: pooled, fixed, random, and between effects; individual or time-based effects; instrumental variable models; variable coefficient models; and general method of moments (GMM) models.

\subsubsection{Syntax}
\begin{verbatim}
> z.out <- zelig((Y ~ X1 + X2, model = "plm", data = mydata, 
+ pan.model="within", pan.index=NULL) 
> x.out <- setx(z.out) 
> s.out <- sim(z.out, x=x.out)
\end{verbatim}

The \texttt{plm} package uses a special \texttt{data.frame()} method, the \texttt{pdata.frame()}, to set section and time index variables. You can pass those indices to your model if you have not set them already using the \texttt{pdata.frame()} command on your data.

\subsubsection{Examples}
Following the examples provided by Croissant and Millo:
\begin{enumerate}
\item Basic estimation of a random effects model with first differences
%\begin{verbatim}

Attach data:

\texttt{> data(Grunfeld, package="plm")}

Estimate model with Zelig:

\begin{verbatim}
> z.plm1 <- zelig(inv ~ value + capital, model="plm", data=Grunfeld, 
+ pan.model="random", pan.index=c("firm","year"))
\end{verbatim}

Summarize the regression output:

\texttt{> summary(z.plm1)}

Take a first difference by setting explanatory variables to their default (mean/mode) values, with high (75th percentile) and low (25th percentile) values for the specified explanatory variable:

\begin{verbatim}
> x.high <- setx(z.plm1, value = quantile(Grunfeld$value, 0.75))
> x.low <- setx(z.plm1, value = quantile(Grunfeld$value, 0.25))
\end{verbatim}

\begin{verbatim}
> s.plm1 <- sim(z.plm1, x=x.high, x1=x.low)

> summary(s.plm1)

> plot(s.plm1)
\end{verbatim}


\item Zelig, plm, and multiple imputation

If you plan to use Zelig to analyze TSCS data in multiple imputation datasets, please use the following syntax:
\begin{verbatim}
> z.out <- zelig(log(gsp) ~ log(pcap) + log(pc) + log(emp) + unemp, 
+ model="plm", data=mi(produc1,produc2,produc3,produc4,produc5,produc6), 
+ plm.index=c("state", "year"))
\end{verbatim}
That is, set the sectional/individual and time indices in the \texttt{zelig()} command rather than making each MI dataset a pdata.frame before running zelig.
Setting the multiply imputed data frames as \texttt{pdata.frames} before using the \texttt{zelig()} command generates the following error and warning:

\texttt{> Error in model.matrix.pFormula(formula, data, rhs = 1, model = model,  : NA in the individual index variable}
  
\texttt{> In `[.data.frame`(index, as.numeric(rownames(mf)), ) : NAs introduced by coercion}
  
\end{enumerate}


\subsubsection{Model}
Let \(Y_{it}\) be the continuous dependent variable, with group (cross-sections) \(i\) and time periods \(t\), for \(i=1,\ldots,J\) and \(t=1,\ldots,T\)

Then we may estimate the following model (known as the ``unobserved effects model'')
\begin{eqnarray*}
y_{it}	&	=	&	\alpha + \beta_{itk}x_{itk} + u_i + \epsilon_{it}
\end{eqnarray*}
where \(\beta_k\) are the explanatory parameters, \(u_i\) are the unobserved cross-sectional effects, and \(\epsilon_{it}\) is the idiosyncratic error.

We often discuss whether to treat the unobserved effects model as one of ``random'' effects or ``fixed'' effects. In practice, this means:
\begin{itemize}
\item 	A \emph{random effects model} assumes that there is no correlation between the observed explanatory variables and the unobserved effect, \(u_i\). That is, \(\mathrm{Corr}(\beta_{it},u_i) = 0\). The random effects model estimates \(u_i\) and \(\epsilon_{it}\) as a \emph{composite error term}, \(v_{it}\).
\item	A \emph{fixed effects model} allows for some arbitrary correlation. That is, \(\mathrm{Corr}(\beta_{it}, u_i) \neq 0\). The fixed effects model estimates the component error terms \(\epsilon_{it}\) and \(u_i\) \emph{separately}. (\(u_i\) is, in essence, estimated as a vector of intercepts, one for each \(i\) in the regression equation.) Fixed effects models, in essence, transform each variable by subtracting the group mean of each variable (the ``fixed effect'' of being in the group). (Thus, there is no overall estimation intercept in the results of these estimations.)
\end{itemize}

The model's components are those of least squares regression.
\begin{itemize}
\item The \emph{stochastic} component is described by the densities % with i.i.d. distributions (which are generally treated as Normal)
\begin{eqnarray*}
%Y_i		&	=	&	Normal(y_i|\mu_i, \sigma^2)
Y_{it}	&	=	&	f(y_{it}|\mu_{it}, \sigma^2) \\
u_i		&	=	&	g(\mu_u, \sigma^2_u)\\
\epsilon_{it}	&	=	&	h(\mu_\epsilon, \sigma^2_\epsilon)
\end{eqnarray*}



\item The \emph{systematic} component is given by
\[\mu_{it} = x_{it}\beta\]
\end{itemize}


\subsubsection{Quantities of Interest}
\begin{itemize}
\item 
The expected value is given by the mean of simulations from the stochastic component:
\[E(Y_{it}) = X_{it}\beta\]
\item The predicted value is drawn from the distribution defined by 
\[\mathrm{Normal}(\mu_{it}, \sigma^2)\]
\item The first difference in expected values, given when \(x1\) is specified is
\[FD = E(Y|x_1) - E(Y|x)\]
\end{itemize}


\subsubsection{Output Values}
\begin{itemize}
\item From the {\tt zelig()} output stored in {\tt z.out}, you may
extract:
\begin{itemize}
\item	\texttt{coefficients}
\item	\texttt{residuals}
\item	\texttt{df.residual}
\item	\texttt{ercomp}
\item	\texttt{args}
\end{itemize}
\item From {\tt summary(z.out)}, you may extract:
\begin{itemize}
\item	\texttt{coefficients}
\item	\texttt{vcov}
\item	\texttt{ercomp}individual and idiosyncratic variance, std. error, shares of each, and theta\footnote{(Random effects models only) theta is a function of the error component variances, \(\sigma^2_u\) and \(\sigma^2_\epsilon\), and is used in the GLS transformation required to estimate RE coefficients. It weights the estimates of the coefficients as follows:
\begin{eqnarray*}(y_{it}-\theta\overline{y}_i)	&	=	& (1-\theta)\alpha + (x_{it}-\theta\overline{x}_i)\beta + \{(1-\theta)u_i + (\epsilon_{it}-\theta\overline{\epsilon}_i)\} \\
\theta	&	=	&	1- \sqrt{\frac{\sigma^2_\epsilon}{T\sigma^2_u+\sigma^2_\epsilon}}
\end{eqnarray*}
where the \(\sigma^2\) are either given or estimated from the data.
}
\item	\texttt{fstatistic}
\item	\texttt{r.squared} both raw and adjusted
\item	\texttt{} 
\end{itemize}
\item From the {\tt sim()} output stored in {\tt s.out}:
\begin{itemize}
\item	\texttt{qi\$ev}: the simulated expected values for the specified values of x.
\item	\texttt{qi\$pr}: the simulated predicted values drawn from the distributions defined by the expected values.
\item	\texttt{qi\$fd}: the simulated first differences (or differences in expected values) for the specified values of x and x1.
\end{itemize}
\end{itemize}


\subsubsection{Further Information}

This Zelig module does not implement the %\texttt{pht()} (Hausman-Taylor model for instrumental variable estimation), \texttt{pvcm()} (variable coefficients models for panel data), \texttt{pgmm()} (generalized method of moments estimation for panel data), or 
\texttt{pggls()} (general FGLS estimators for panel data). Users needing such functions may wish to investigate the mixed-effects regression procedures in Zelig.

Linear models for panel (TSCS) data and the sample data used in the examples above are a part
of the \texttt{plm} package by \citeauthor{croissantmillo2008a}. Users will wish to refer to the applicable help files: \texttt{help(plm)}, \texttt{help(pht)}, \texttt{help(pvcm)}, and \texttt{help(pgmm)}, as well as \citet{croissantmillo2008a}. 

For more general help on panel data techniques, three excellent treatments are (in increasing order of difficulty and technical detail): \citet{wooldridge2006, greene2007a, wooldridge2002}.

\subsubsection{Contributors}
You may cite the panel data models in Zelig as:


@article{croissantmillo2008a,
	Accepted = {2008-03-20},
	Author = {Croissant, Yves and Millo, Giovanni},
	Coden = {JSSOBK},
	Day = {29},
	Issn = {1548-7660},
	Journal = {Journal of Statistical Software},
	Month = {7},
	Number = {2},
	Pages = {1--43},
	Submitted = {2007-06-06},
	Title = {Panel Data Econometrics in R: The plm Package},
	Url = {http://www.jstatsoft.org/v27/i02},
	Volume = {27},
	Year = {2008},
	Bdsk-Url-1 = {http://www.jstatsoft.org/v27/i02}}
	

@book{wooldridge2006,
	Address = {Mason, OH},
	Author = {Wooldridge, Jeffrey M.},
	Booktitle = {Introductory econometrics: a modern approach},
	Call-Number = {HB139},
	Dewey-Call-Number = {330.01/5195},
	Edition = {3rd ed},
	Genre = {Econometrics},
	Isbn = {0324289782 (hbk.)},
	Library-Id = {2005924660},
	Publisher = {Thomson/South-Western},
	Title = {Introductory econometrics: a modern approach},
	Year = {2006}}


@book{wooldridge2002,
	Address = {Cambridge, Mass.},
	Author = {Wooldridge, Jeffrey M.},
	Call-Number = {HB139},
	Dewey-Call-Number = {330/.01/5195},
	Genre = {Econometrics},
	Isbn = {0262232197 (cloth)},
	Library-Id = {2001044263},
	Publisher = {MIT Press},
	Title = {Econometric analysis of cross section and panel data},
	Year = {2002}}

@book{greene2007a,
	Address = {Upper Saddle River, New Jersey},
	Author = {Greene, William H.},
	Edition = {6th},
	Publisher = {Prentice Hall},
	Title = {Econometric Analysis},
	Year = {2007}}
